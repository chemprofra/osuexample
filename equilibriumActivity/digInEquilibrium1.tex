\documentclass{ximera}
%% handout
%% space
%% newpage
%% numbers
%% nooutcomes

%%\input{../preamble} %% we can turn off input when making a master document

\outcome{Introduce equilibrium.}
\outcome{Reaction arrows.}
 
\title{Equilibrium}

\begin{document}
\begin{abstract}
Here, we discuss the conditions of equilibrium.
\end{abstract}
\maketitle


Under certain conditions, an equilibrium reaction will appear to remain unchanged.  The reactant and product concentration are not changing. That is, the reaction of the reactants combining to form products occurs at the same time as products decomposing back into reactants. So, the net concentration of these species are constant. We call this a dynamic equilibrium system because the reactions continue even when there appears to be no change. 

In an everyday situation, this is analogous to people both entering and exiting a store. Some people can walk through an entryway, while at the same time others leave through an exit. In a situation where an equal number of people enter the store as those that leave the store, the net number of people inside the store remains constant. While different people may be inside, the total number of people is the same. To a casual observer, there seems to be no change, despite the continual entering and leaving. 

This same situation can happen in a chemical system. Take, for instance, the pink/blue reaction system described previously,

%
%
     $\mathrm{\textcolor{blue}{CoCl_{4}^{ 2-}}+ 6H_2O  \rightleftharpoons \textcolor{magenta}{Co(H_2O)_6^{ 2+}} + 4Cl^-}$


The forward and reverse reactions both continue to occur at equilibrium. At equilibrium, there would be no apparent change in the color because the 
$\mathrm{CoCl_{4}^{ 2-}}$ reacting with water to form the pink and $\mathrm{Cl^-}$ happens at the same speed as the pink compound degrading back to blue 
$\mathrm{CoCl_4^{2-}}$. Hence, there is no net change in overall color. So, we say the reactions are occurring at the same rate(or speed), which results in no net change.

\begin{question}
Consider the following reaction:

$\mathrm{A+B \rightleftharpoons C}$


At equilibrium, there is no change in the concentration of C, but the concentrations of A and B do change.

 \begin{multipleChoice}
	\choice{Yes}
	\choice[correct]{No}
	\choice{Yes, but only at low temperature}
\end{multipleChoice}
\end{question}

\begin{question}
Consider the (same) following reaction:

$\mathrm{A+B \rightleftharpoons C}$


Do the forward and reverse reactions continue when equilibrium is reached.

 \begin{multipleChoice}
	\choice[correct]{Yes}
	\choice{No}
	\choice{Yes, but only at low temperature}
\end{multipleChoice}
\end{question}

\begin{question}
Which of the following are correct at equilibrium? (pick all that apply)
%
\begin{selectAll}
	\choice[correct]{The concentration of A does not change.}
	\choice{The reverse reaction stops.}
	\choice[correct]{The forward reaction continues.}
	\choice{The concentration of A increases.}
	\choice[correct]{The concentration of B stays the same.}
\begin{hint}
Think about what happens, at the molecular level, at equilibrium.
\end{hint}
 \end{selectAll}
\end{question}



\end{document}
