\documentclass{ximera}
%% handout
%% space
%% newpage
%% numbers
%% nooutcomes

%%\input{../preamble} %% we can turn off input when making a master document

\outcome{Introduce the Equilibrium Constant Expression}
\outcome{Understand equilibrium using the Law of Mass Action.}
 
\title{Equilibrium}

\begin{document}
\begin{abstract}
Here, we discuss the Equilibrium Constant Expression.
\end{abstract}
\maketitle

At equilibrium, we have seen that the rates of forward and reverse reactions are the same. So, while the rates are the same for the reactions, the concentrations of the reactants and products are different. In fact, the concentrations of two (or more) reactants in a reaction need not be equal; same goes for the products.

One of the ways to quantitate the ratio of the species of the reactants and products is using the Law of Mass Action. This law gives a numerical value for a given reaction. 


%
%
    $\mathrm{aA+ bB  \rightleftharpoons cC + dD}$
                                                                         

where A, B, C and D are hypothetical chemical species. The lower case letters are corresponding stoichiometric coefficients. The ratio of products to reactants is given by:

$\mathrm{K_c=\frac{[C]^c[D]^d}{[A]^a[B]^b}}$

The $\mathrm{K_c}$ is called the equilibrium constant. The entire mathematical formula is called the Equilibrium Constant Expression. 

\begin{question}
For the following reaction:

$\mathrm{N_2+ 3H_2  \rightleftharpoons 2NH_3}$


The equilibrium constant expression is:

 \begin{multipleChoice}
	\choice{$\mathrm{K_c=\frac{[NH_3]^2}{[N_2+H_2]}}$}
	\choice[correct]{$\mathrm{K_c=\frac{[NH_3]^2}{[N_2] [H_2]^3}}$}
	\choice{$\mathrm{K_c=\frac{[N_2] [H_2]^3}{[NH_3]^2}}$}
	\choice{$\mathrm{K_c=\frac{[N_2] + [H_2]^3}{[NH_3]^2}}$}
\end{multipleChoice}
\end{question}





\end{document}
