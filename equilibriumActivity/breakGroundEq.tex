\documentclass{ximera}
%% handout
%% space
%% newpage
%% numbers
%% nooutcomes

%%\input{../preamble} %% we can turn off input when making a master document

\outcome{Introduce equilibrium.}
\outcome{Reaction arrows.}
 
\title{Equilibrium}

\begin{document}
\begin{abstract}
Two chemists discuss reversible reactions.
\end{abstract}
\maketitle




Check out this dialogue between two chemistry students. 

Student 1: My professor showed this incredible demonstration in class today. He created a flame that was green and, it was a tornado flame!

Student 2: I love demos!

Student 1: The professor said the fire resulted from this reaction:

        $\mathrm{CH_3OH + O_2 \rightarrow CO_2 + H-2O}$

    also, the tornado resulted from the air current. 

Student 2: That tornado flame is cool!

Student 1: I know. I was thinking, though, could we get the $\mathrm{CO_2}$ and $\mathrm{H_2O}$ to go back to $CH_3OH$ an $O_2$?

Student 2: Not likely. I think that is an irreversible reaction. 

Student 1: Irreversible?

Student 2: Yeah. You can?t easily get the reactants back. However, some reactions are reversible. 

Student 1: Tell me more.

Student 2: My lecture professor showed this demo in class:

    $\mathrm{\textcolor{blue}{CoCl_{4}^{ 2-}}+ 6H_2O  \rightleftharpoons \textcolor{magenta}{Co(H_2O)_6^{ 2+}} + 4Cl^-}$
   

    He was able to change the colors back and forth.

Student 1: Cool colors. So, that means this reaction is reversible.

Student 2: Yes. This is an equilibrium reaction where the forward and reverse reactions both continue.

Student 1: Good point. Does this work with all equilibrium reactions?


\begin{question}
Which reaction arrows represent an equilibrium reaction? (pick all that apply)
%
\begin{selectAll}
	\choice{$\leftarrow$}
	\choice[correct]{$\rightleftharpoons$}
	\choice{$\leftrightarrow$}
	\choice[correct]{$\leftrightharpoons$}
	\choice{$\Leftarrow$}
\begin{hint}
Think about the arrow!
\end{hint}
 \end{selectAll}

\end{question}

\begin{question}
Does this reaction system will go in ?reverse??
%
 $\mathrm{CH_{3}COOH \rightleftharpoons CH_{3}COO^- + H^+}$
 \begin{multipleChoice}
	\choice[correct]{Yes}
	\choice{No}
	\choice{I don't know}
\end{multipleChoice}
\end{question}







\end{document}
